\chapter{Introduction}
Computer networks and systems have turned out to be vital instruments for current business.
The amount of data that is stored and accessed by users throughout the world have been increased tremendously during last one decade.
According to Cisco's Visual Networking Index initiative \cite{cisco}, it is estimated that, before the end of 2016, worldwide Internet traffic will achieve 1.1 zettabytes per year and by 2019, global traffic is anticipated to hit 2 zettabytes for every year.
Much of this information is to some degree confidential and its protection is required.
As the number of connected nodes on the Internet has been increased, the various malicious attempts to gain control of the network is also increased.
Attackers use different networking tools like IP spoofing and Internet wide scanning to understand the vulnerabilities in the network.
So it is essential to build network security monitor devices to help uncover attempts by attackers to gain unauthorized access to computer networks.
This results to gain easy access to data stored in devices\\\\
Internet wide scanning is a robust technique used by researchers to study and evaluate Internet.
It has become popular in research field to learn new types of vulnerabilities, to track many risk mitigation activities.
However such a powerful tool could be used by attackers to locate  opened/vulnerable ports.
So it is important to do a systematic study on how these scanners behave by analysing the incoming connection requests from them.
This can be achieved using a Network telescope \cite{caida} which captures received packets and analyse it.\\\\
Network telescopes (NT) permit the capture of illegitimate data on a wider scale.
In general, NT captures the incoming traffic from an unutilised or dark Internet address space.
Network telescope is capable of detecting large scale of events with respect to the address spaces it monitors.
If an NT is being assigned with more Internet address spaces, its resolution will be also high \cite{irvin}.


% 	\section{Research Question}
% 		%  1. a concise statement of the question that your thesis tackles 
% 		%  2. justification, by direct reference to section 3 that your question is previously unanswered
% 		%  3. discussion of why it is worthwhile to answer this question.
	
	\section{Problem Statement}
	The volume of backscatter or Internet Background Radiation (IBR) \cite{pang2004characteristics}\cite{wustrow2010internet} has been increased excessively with the growth of Internet after its commercialisation in mid 1990s.
	Despite of the fact that, this is partly due to the system misconfiguration, a significant portion of such traffic can be seen as the result of either deliberate attack or network scanning tools.
	Moreover there has been an increase in spreading of malicious software or malware during recent years.
	Thereby the volumes and intensity of what is deemed to be traffic with malicious intent arriving at the Internet hosts are also increased.\\\\
	While Internet scanners can be used for many research purposes, there is a high chance of mistreating their high potential by attackers to enumerate the vulnerable ports in the Internet.
	The aftermath of all these attacks is that systems and organisations exposed to the Internet are facing a greater level of risk than they had before.\\\\
	Aiming at Internet scanner's detection, we developed a small network telescope. 
    Packet capture took place during two different schedules.
    We initially used active network telescope to capture the packets assuming that all the packets received by the network telescope are illegitimate packets.
    The active network telescope is called so since it is capable of receiving the incoming traffic and just responds to it.
    However, our configuration was designed in such a way that only port 22 was open and able to respond to the connection requests for accessing the dataset through SSH service.
    The remaining ports were closed during the first stage and rejecting any incoming connection requests.
	Since proposed system consists of limited IP address space, we later combined the initial system with honeypot to exploit the received connections by sending vulnerable replies.
	The honeypot was setup in such a way that it responds and process the requests to entire TCP network range and some UDP services with a legitimate service-like behavior.
	The main goal of the thesis is to analyse the received packets from the two previously mentioned setups and check if it is feasible to identify a pattern depending on the behavior of these scanners and transport layer protocols used to send the packets.
	Moreover it is also important to compare the obtained results of the initial configuration with the latter structure and check if the behavior has been changed.
	
    
	